\chapter*{Abstract}

This project tests two methods for landmark detection on the seabed using side scan sonar (SSS). It provides an overview of the advantages and drawbacks of the two methods for finding the most promising method to use in a collaborative feature-based Simultaneously Localization and Mapping (SLAM) pipeline. Robust landmark detection that is invariant to changing environments is an enabling factor for doing feature-based SLAM using an inexpensive SSS and sensor stack on AUV’s. Current methods primarily rely on deep learning or classical methods such as shadow, echo, or edge detection. However, the research available does not extensively test different methods against each other, and it is therefore difficult to compare them against each other. To address this, this paper tests two classical methods on SSS data collected from various locations and qualitatively compares them against each other. Using peak detection, the first method finds shadows and echoes in 1D sonar scanlines. The second method finds shadows in 2D sonar images using expert rules. The results show that both methods need to be more robust and invariant. The 1D method does not provide any means of combining information from different scanlines, making the method detect the same feature on the seabed as different landmarks over several scanlines. The 2D method, on the other hand, already works in 2D and does not have this problem. However, it lacks performance where one landmark is split into several shadows and does not appear as one continuous shadow. Future research should focus on making new methods or improving on existing methods such that they work in the cartesian space, making, for example, parameters invariant to changes in sonar range and sonar resolution. Further, an openly available benchmark dataset would give grounds for comparing all new methods against a common ground. 