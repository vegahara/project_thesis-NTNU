\chapter*{Abstract}

This report tests two landmark detection methods using side scan sonar (SSS) and presents a novel quality indicator for assessing SSS data in terms of its information content. The report moreover provides an overview of the advantages and drawbacks of the two methods and strives for finding which one of the methods is the most promising for the purpose of implementing a feature-based Simultaneous Localization and Mapping (SLAM) pipeline. In the process, the report comments on the fact that robust landmark detection (i.e., detection that is invariant to changing environments) is an enabling factor for obtaining accurate feature-based SLAM. 

For this, we note that current methods primarily rely on deep learning or classical strategies such as shadow and/or echo detection. However, to the best of our knowledge, the available research does not extensively test different methods against each other; it is therefore difficult to judge what is best when. To address this, this report tests two classical methods on SSS data collected from the Trondheim fjord and qualitatively compares them against each other. In more detail, the first method uses peak detection to find shadows and echoes in 1D sonar swaths. The second method instead finds shadows in 2D sonar images using expert rules. The results show that both methods lack robustness and, to a varying degree, cannot consistently detect landmarks.

For the purpose of carrying on the analysis above in the most rigorous way possible, we propose a dedicated quality indicator that uses the raw information available from the sensors to assess the quality of the data they have been producing. This quality indicator is based on assessing the amount of overlapping of consecutive swaths and does so for the ground range at the maximum sensor opening. Therefore, the proposed index only reports on overlapping in the ranges we expect to find valuable information. However, it does not directly consider changes in speed and angles and thus is not able to detect (and thus indicate) poor quality when anomalies in the data occur from changing speed or roll angle. 
