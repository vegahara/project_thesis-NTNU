\chapter{Background}

\section{Simultaneous localization and mapping}

Simultaneous localization and mapping (SLAM) is a methodology of building a map of an unknown environment as a mobile platform explores it and, at the same time, localizes the mobile platform in the same map. 
% End with front/back-end.

The front-end of SLAM consists of feature extraction and data association.

The back-end of SLAM mainly consists of maximum a posteriori (MAP) estimation that smooths the map and the poses of the mobile platform. 

Factor graphs are the de-facto standard for modeling the SLAM methodology. 

\section{Side Scan Sonar}

The side scan sonar is a two-transducer sonar mounted so that each transducer points downwards and outwards to the port and starboard, respectively.

\section{Landmark Detection}

Landmark detection is to pick out interesting features, or in other words, landmarks, from data generated from observing the surroundings.

\subsection{Landmark detection using classical methods}

Classical landmark detection methods for side scan sonar data exploit signal and image processing methods such as peak, edge, and shadow detection. 

\subsubsection{Detecting shadows and echoes in sonar scan lines}

Finding signal peaks and the width and prominence of the peaks are essential metrics in 1D signal processing. 

\subsubsection{Data normalization and smoothing}

Cubic splines can be used to both normalize SSS data and to smooth the data to remove noise. 

\subsubsection{Generating consistent landmarks}

Morphological operators or low-pass filters can be utilized to ensure consistent landmarks and that one landmark is generated for each natural landmark.

\subsubsection{1D landmark detection}

1D landmark detection does the detection in the 1D scan lines, for example, by finding peaks and valleys in the scan line corresponding to echoes and shadow, respectively.

\subsubsection{2D landmark detection}

2D landmark detection works in a 2D sonar image and uses techniques such as intensity thresholding, edge detection, and general image processing.

\subsection{Landmark detection using deep learning}

In recent years, making use of the rapidly evolving field of deep learning to do landmark detection has grown in popularity.








% This is not part of the introduction, just old notes.

\section{Background}

\subsection{Autonomous underwater vehicles}

AUVs depend upon dead reckoning for navigation because of the lack of global position measurements from, for example, GNSS.
% Avslutt avsnittet med å si at SLAM er state of the art etc og pek videre på neste subsection.

\subsection{Simultaneously Localization and Mapping}

SLAM is the state-of-the-art framework for robot navigation and localization.

Feature-based or indirect SLAM uses features or landmarks to help the robot navigate.
% Avslutt med at landmarks finnes gjennom sensoren som er farkostens "øyne" og for AUVer er det veldig typisk en sonar. 

\subsection{Sonar}

A sonar is a sensor that sends out acoustic pulses and measures the intensity of the echo return, and in such a way, can provide a representation of the surroundings.

Under the supervision of  Damiano  Varagnolo and  Simon  Andreas  Hagen  Hoff, Bjørnar Reitan Hogstad has developed a sonar processing pipeline that forms the basis for this project. (Describe earlier work)

\subsection{Landmark Detection}

A landmark detector that can robustly detect landmarks in the seabed over various environments is needed to enable feature-based SLAM.





