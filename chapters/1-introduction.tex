\chapter{Introduction}

As the population grows, utilizing the resources and areas in and on the ocean becomes more critical, as resources and land space are limited and in great demand. It is expected a large increase in both green energy and aquaculture production in the future \cite{Oceans2050}. Recent examples are the offshore wind farms Hywind Tampen \cite{HywindEquinor} and the offshore fish farming project Ocean Farm 1\cite{HavbasertASA}. Deep sea mining is also an up-and-coming market, with the potential to provide rear-earth for producing, for example, electronic devices and vehicles \cite{Bogue2015UnderwaterApplications}.

To make use of the ocean, one crucial aspect is monitoring the fragile ecosystems and the installed infrastructure. The ecosystems are essential to monitor, ensuring that our activity in the sea is sustainable and does not destroy fragile ecosystems. Fish farming in the fjords has proven to put the ecosystems out of balance; for example, how salmon lice threaten the wild salmon swimming in the same area as the fish farms and how the waste from fish farming affects the ecosystem in the ocean \cite{Fiskeoppdrett}. Further, there have been worries about the impact on the ecosystems deep sea mining can have \cite{UnderstandingTechnology}. On the other side, monitoring and maintaining the installed infrastructure is crucial from an economic perspective. A typical operation is inspecting subsea pipelines and cables, ensuring they have not moved and are in good shape. 

Today, AUVs are used for monitoring and surveying the ocean \cite{Nicholson2008TheTechnologies}. They are often more cost-effective than manned vessels and, in addition, more environmentally friendly. Even though they are used in commercial applications, they still have some challenges constraining their operational areas. Such challenges include underwater navigation and communication, making long-time operations without surfacing challenging. 

One of the most significant challenges for doing long-time AUV operations is that electromagnetic waves have constraints in their use underwater, making GNSS unavailable and, in practice restricting communication to acoustic communication with low bandwidth for submerged AUVs \cite{Nicholson2008TheTechnologies}. This makes long-term autonomous operations difficult. The restriction of communication restricts the possibilities for monitoring and human supervision, making the AUV more dependent on its autonomous capabilities. One key element for this is reliable and accurate navigation and localization. Without GNSS, reliable localization either needs expensive inertial sensors or an external position system, such as acoustic beacons for localization, that is expensive to install and restricts the operating area. A promising solution would be simultaneous localization and mapping (SLAM).

SLAM has significantly impacted mobile robotics, improving the localization and mapping of unknown environments. It has, for example, been applied to drones \cite{VonStumberg2017FromExploration}, where a camera is typically used to provide the perception of the environment. \cite{Hidalgo2015ReviewTechniques}This could be a solution for AUVs as well. However, using a camera subsea is problematic because of the turbidity in the water and the lack of light at greater depths, significantly reducing sight. 

Another way of getting a perception of the surroundings is to use sonar, a sensor that uses acoustics for sensing its surroundings. A sonar transmits acoustic pulses and measures the echoes from its environment to give a perception of the environment. A great advantage over cameras is that it is not affected by turbidity and light conditions, making sonar an excellent alternative for underwater applications. Because of this, new research focuses on SLAM using sonars to perform long-time operations without additional measures.

\section{Project Description}

This project aims to test two seabed landmark detection methods using side scan sonar (SSS). It provides an overview of the advantages and drawbacks of the different techniques for finding the most promising method to use in a collaborative feature-based Simultaneously Localization and Mapping (SLAM) pipeline.

The work has been done under the supervision of Simon Andreas Hagen Hoff and Damiano Varagnolo, and their insight and help have been greatly appreciated. Thanks to Tore Mo-Bjørkelund and Ambjørn Waldum for providing side scan sonar data and insight into how the data was acquired. 


\section{Report structure}

First, this report will present the background needed for context and understanding of the work in this report. Secondly, the landmark detectors and the quality indicator will be presented together with an explanation of how the results are considered. Thirdly, the results of running the landmark detectors and the quality indicator on side scan sonar data are presented. Last, a discussion about the landmark detectors and quality indicators, their performance, weaknesses, and strengths, together with a comparison of the two landmark detectors.  


