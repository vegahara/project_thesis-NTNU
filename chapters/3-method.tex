\chapter{Method}

\section{1D Landmark Detector using Peak Detection}

The 1D landmark detector does peak detection in scan lines to find shadows and echoes and filter out non-significant echoes and shadows using a simple rule.

To improve upon the 1D landmark detector, as \todo{cite 1D paper} suggests, normalized data is used instead of raw sonar data. 

To further improve the method, we reject the landmarks in scan lines where more than $x\%$ of the bins are considered detected landmarks as false positives. 

\section{2D Landmark Detector using Expert Rules}

The 2D landmark detector finds shadows by intensity thresholding and uses a set of rules to filter out the shadows of the right size and form. 

Instead of the suggested intensity threshold in \todo{cite 2d paper}, this report introduces a new, more robust intensity threshold.

We perform low-pass filtering and a closing operation to improve robustness and landmark consistency in the sonar image.

\section{Qualitative comparison}

 This report makes a qualitative comparison of the landmark detectors' results to compare them against each other.

To compare the method, an understanding of what parts of the sonar image are actual landmarks and what properties are consistent with a landmark. 

Landmark consistency is an essential qualitative measure and tells to what extent a landmark detection method can detect one real landmark as one coherent landmark or several smaller landmarks. 

Another vital measure when comparing landmark detection methods is the sensitivity and complexity of the tuning.



% Skrive litt om hvordan jeg går fra sonar data til bilder jeg bruker.